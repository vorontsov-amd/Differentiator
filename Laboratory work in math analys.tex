\documentclass[a4paper,12pt]{article} % добавить leqno в [] для нумерации слева
\usepackage[a4paper,top=1.3cm,bottom=2cm,left=1.5cm,right=1.5cm,marginparwidth=0.75cm]{geometry}
\usepackage{cmap}					% поиск в PDF
\usepackage[warn]{mathtext} 		% русские буквы в фомулах
\usepackage[T2A]{fontenc}			% кодировка
\usepackage[utf8]{inputenc}			% кодировка исходного текста
\usepackage[english,russian]{babel}	% локализация и переносы
\usepackage{physics}
\usepackage{multirow}
\allowdisplaybreaks

%%% Нормальное размещение таблиц (писать [H] в окружении таблицы)
\usepackage{float}
\restylefloat{table}



\usepackage{graphicx}

\usepackage{wrapfig}
\usepackage{tabularx}

\usepackage{hyperref}
\usepackage[rgb]{xcolor}
\hypersetup{
	colorlinks=true,urlcolor=blue
}

\usepackage{pgfplots}
\pgfplotsset{compat=1.9}

%%% Дополнительная работа с математикой
\usepackage{amsmath,amsfonts,amssymb,amsthm,mathtools} % AMS
\usepackage{icomma} % "Умная" запятая: $0,2$ --- число, $0, 2$ --- перечисление

%% Номера формул
\mathtoolsset{showonlyrefs=true} % Показывать номера только у тех формул, на которые есть \eqref{} в тексте.

%% Шрифты
\usepackage{euscript}	 % Шрифт Евклид
\usepackage{mathrsfs} % Красивый матшрифт

%% Свои команды
\DeclareMathOperator{\sgn}{\mathop{sgn}}

%% Перенос знаков в формулах (по Львовскому)
\newcommand*{\hm}[1]{#1\nobreak\discretionary{}
	{\hbox{$\mathsurround=0pt #1$}}{}}

\date{\today}

\usepackage{gensymb}

\begin{document}

\begin{titlepage}
	\begin{center}
		{\large МОСКОВСКИЙ ФИЗИКО-ТЕХНИЧЕСКИЙ ИНСТИТУТ (НАЦИОНАЛЬНЫЙ ИССЛЕДОВАТЕЛЬСКИЙ УНИВЕРСИТЕТ)}
	\end{center}
	\begin{center}
		{\large Физтех-школа радиотехники и компьютерных технологий}
	\end{center}
	
	
	\vspace{4.5cm}
	{\huge
		\begin{center}
			{\bf Отчёт о выполнении лабораторной работы 2.2.8}\\
			Практика взятия производной сложной фукции
		\end{center}
	}
	\vspace{2cm}
	\begin{flushright}
		{\LARGE Автор:\\ Воронцов Амадей Александрович \\
			\vspace{0.2cm}
			Б01-102}
	\end{flushright}
	\vspace{8cm}
	\begin{center}
		Долгопрудный\\
		\today
	\end{center}
\end{titlepage}

\section{Дисклеймер}
Данная статья не рекомендуется к прочтению лиц младше 18 лет, либо не сдавшим коллоквиум студентам первого курса. Приведенная информация является сугубо мнением автора. Данная статья может задеть ваши чувства, связанные с этикой и нормами вашей жизни, поэтому просим всех гуманнитариев убрать глаза от экрана. 

\section{Введение}
\textbf{Цель работы:}  \begin{enumerate}
	\item изучить алгоритм взятия производной сложной функции
	\item бесполезно потратить своё время
\end{enumerate}

\textbf{В работе используются:} ручка, бумага, мозги
\section{Теоретические сведения}
\textbf{Определение:} Производной функции $f$ называется предел $ \lim_{x\to\infty} \frac{f(x) - f(x_0)}{x - x_0}$, если этот предел конечен, либо равен $+ \infty \text{ или} - \infty$. Обозначается этот предел $f'(x_0)$.
\newline
\textbf{Теорема 1:} Пусть функции $f$ и $g$ имеют конечные производные в точке $x_0$. Тогда функции $f + g, f \cdot g, \frac{f}{g}$ имеют производные в точке $x_0.$ (В последнем случае нужно требовать, чтобы $g'(x_0) \neq 0 $), причем в точке $x_0$ выполняются равенства

$$    \left( f(x) + g(x) \right)' = f'(x) + g'(x) $$
$$    \left( f(x)g(x) \right)' = f'(x)g(x) + f(x)g'(x) $$
$$    \left( \frac{f(x)}{g(x)} \right)' = \frac{f'(x)g(x) - f(x)g'(x)}{g^2(x)} $$
\newline
\textbf{Теорема 2:} Пусть функция $f$ имеет конечную производную в точке $x_0$, а функция $g$ имеет конечную производную в точке $u_0 = f(x_0)$. Тогда функция $h(x) = g(f(x))$ имеет производную в точке $x_0$, причем $h'(x_0) = g'(u_0)f'(x_0)$

\section{Ход работы}

\textbf{Задача:} найти 
$ \left(    \sin \left( x\right)^{ \cot \left(   6  x^{ \sinh \left( x\right)}\cdot  4\right)}\cdot \left(    3  x^{ 2}\cdot  2- \frac{  2^{ x}}{ x}\right)+ 1 \right)'$ 
 \newline 
\text{Как легко видеть, } $ \left| \oplus_{k \in S} \left( \mathfrak{K}^{F^{\alpha}(i)} \right) \right| \preceq \mathfrak{N}_1 \text{ при } {\lfloor \mathfrak{H} \rfloor}_W \cap F^{\alpha}(\mathbb{N}) \neq \varnothing \text{ поэтому } $\newline $$ \left(    \sin \left( x\right)^{ \cot \left(   6  x^{ \sinh \left( x\right)}\cdot  4\right)}\cdot \left(    3  x^{ 2}\cdot  2- \frac{  2^{ x}}{ x}\right)+ 1 \right)' =  \left(   \sin \left( x\right)^{ \cot \left(   6  x^{ \sinh \left( x\right)}\cdot  4\right)}\cdot \left(    3  x^{ 2}\cdot  2- \frac{  2^{ x}}{ x}\right) \right)' +  \left( 1 \right)'$$
 \newline 
Следующее утверждение представляет собой перефомулировку предложения 5.3.18, которое вытекает из теоремы 7.1.24, основанной на лемме 2.4.14, являющейся следствием из теоремы 7.1.24 и предложение 5.3.18\newline $$ \left( 1 \right)' = 0$$
 \newline 
Очевидно, что\newline $$ \left(   \sin \left( x\right)^{ \cot \left(   6  x^{ \sinh \left( x\right)}\cdot  4\right)}\cdot \left(    3  x^{ 2}\cdot  2- \frac{  2^{ x}}{ x}\right) \right)' =  \left(  \sin \left( x\right)^{ \cot \left(   6  x^{ \sinh \left( x\right)}\cdot  4\right)} \right)' \cdot \left(    3  x^{ 2}\cdot  2- \frac{  2^{ x}}{ x} \right) +  \left(  \sin \left( x\right)^{ \cot \left(   6  x^{ \sinh \left( x\right)}\cdot  4\right)} \right) \cdot \left(    3  x^{ 2}\cdot  2- \frac{  2^{ x}}{ x} \right)'$$
 \newline 
Утвеверждение вытекает из результата приведенного в [1].[1] К. А. Който (ред.) Югославское математическое общество. Заседание 2. 1925. Сборник рукописей внесекционных докладов. Стр 17.\newline $$ \left(    3  x^{ 2}\cdot  2- \frac{  2^{ x}}{ x} \right)' =  \left(   3  x^{ 2}\cdot  2 \right)' -  \left( \frac{  2^{ x}}{ x} \right)'$$
 \newline 
Обоснование выссказывание имеется на стр. 478 в [2].[2] Аут Оф Пейпер. Теория нумераций. М.Каунтер, 1990. 396 с.\newline $$ \left( \frac{  2^{ x}}{ x} \right)' = \frac{ \left(  2^{ x} \right)' \cdot \left( x \right) -  \left(  2^{ x} \right) \cdot \left( x \right)'}{ \left( x \right)^2}$$
 \newline 
Заинтересованный читатель может найти доказательство этого результата на домашней странице У. Узлера: Error 404 Page not found\newline $$ \left( x \right)' = 1$$
 \newline 
Подробное обоснование будет рассмотрена в главе 7 после развития соответствующей теории\newline $$ \left(  2^{ x} \right)' =  \left(  2^{ x} \right) \cdot \ln \left( 2\right) \cdot  \left( x \right)'$$
 \newline 
Глава 7: Ради простоты предположим, что z = 0 (Общий случай рассмотрен в приложении 2)\newline $$ \left(   3  x^{ 2}\cdot  2 \right)' =  \left(  3  x^{ 2} \right)' \cdot \left( 2 \right) +  \left(  3  x^{ 2} \right) \cdot \left( 2 \right)'$$
 \newline 
Приложение 2: Формальное доказательство выходит за рамки данной монографии\newline $$ \left( 2 \right)' = 0$$
 \newline 
Опытный читатель сразу заметит что\newline $$ \left(  3  x^{ 2} \right)' =  \left( 3 \right)' \cdot \left(  x^{ 2} \right) +  \left( 3 \right) \cdot \left(  x^{ 2} \right)'$$
 \newline 
Если мысли сходятся, то они ограничены\newline $$ \left(  x^{ 2} \right)' = 2 \cdot \left( x \right) ^ 1 \cdot  \left( x \right)'$$
 \newline 
Для любого эпсилон больше нуля\newline $$ \left( x \right)' = 1$$
 \newline 
Доказательство следующего отверждения дается читаетелю в качестве упражнения\newline $$ \left( 3 \right)' = 0$$
 \newline 
Давай, дифференцируй, дифференцируй, мы же миллионеры. Ещё функций напридумываем.\newline $$ \left(  \sin \left( x\right)^{ \cot \left(   6  x^{ \sinh \left( x\right)}\cdot  4\right)} \right)' =   \sin \left( x\right)^{ \cot \left(   6  x^{ \sinh \left( x\right)}\cdot  4\right)} \cdot \left( \left( \cot \left(   6  x^{ \sinh \left( x\right)}\cdot  4\right) \right)' \cdot  \ln \left( \sin \left( x\right)\right) +  \frac{ \cot \left(   6  x^{ \sinh \left( x\right)}\cdot  4\right)}{ \sin \left( x\right)} \cdot \left( \sin \left( x\right) \right)' \right)$$
 \newline 
Не опять, а снова! Для закрепления ещё раз посмотрим как решать этот пример. Берем его, и...\newline $$ \left( \sin \left( x\right) \right)' =  \cos \left( x\right) \cdot  \left( x \right)'$$
 \newline 
\newline $$ \left( x \right)' = 1$$
 \newline 
\newline $$ \left( \cot \left(   6  x^{ \sinh \left( x\right)}\cdot  4\right) \right)' = \frac{ \left(   6  x^{ \sinh \left( x\right)}\cdot  4 \right)'}{  \sinh \left(   6  x^{ \sinh \left( x\right)}\cdot  4\right)^{ 2}}$$
 \newline 
\newline $$ \left(   6  x^{ \sinh \left( x\right)}\cdot  4 \right)' =  \left(  6  x^{ \sinh \left( x\right)} \right)' \cdot \left( 4 \right) +  \left(  6  x^{ \sinh \left( x\right)} \right) \cdot \left( 4 \right)'$$
 \newline 
\newline $$ \left( 4 \right)' = 0$$
 \newline 
\newline $$ \left(  6  x^{ \sinh \left( x\right)} \right)' =  \left( 6 \right)' \cdot \left(  x^{ \sinh \left( x\right)} \right) +  \left( 6 \right) \cdot \left(  x^{ \sinh \left( x\right)} \right)'$$
 \newline 
\newline $$ \left(  x^{ \sinh \left( x\right)} \right)' =   x^{ \sinh \left( x\right)} \cdot \left( \left( \sinh \left( x\right) \right)' \cdot  \ln \left( x\right) +  \frac{ \sinh \left( x\right)}{ x} \cdot \left( x \right)' \right)$$
 \newline 
\newline $$ \left( x \right)' = 1$$
 \newline 
\newline $$ \left( \sinh \left( x\right) \right)' =  \cosh \left( x\right) \cdot  \left( x \right)'$$
 \newline 
\newline $$ \left( x \right)' = 1$$
 \newline 
\newline $$ \left( 6 \right)' = 0$$
 \newline\text{Finally:} $$
 \left(    \sin \left( x\right)^{ \cot \left(   6  x^{ \sinh \left( x\right)}\cdot  4\right)}\cdot \left(    3  x^{ 2}\cdot  2- \frac{  2^{ x}}{ x}\right)+ 1 \right)'= $$ \newline $     \sin \left( x\right)^{ \cot \left(  24  x^{ \sinh \left( x\right)}\right)}\cdot \left(  \left( \frac{ -  24\cdot    x^{ \sinh \left( x\right)}\cdot \left(   \cosh \left( x\right) \ln \left( x\right)+ \frac{ \sinh \left( x\right)}{ x}\right)}{  \sin \left(  24  x^{ \sinh \left( x\right)}\right)^{ 2}}\right) \ln \left( \sin \left( x\right)\right)+ \left( \frac{ \cot \left(  24  x^{ \sinh \left( x\right)}\right)}{ \sin \left( x\right)}\right) \cos \left( x\right)\right)\cdot \left(   6  x^{ 2}- \frac{  2^{ x}}{ x}\right)+   \sin \left( x\right)^{ \cot \left(  24  x^{ \sinh \left( x\right)}\right)}\cdot \left(   12 x- \frac{     2^{ x} \ln \left( 2\right)\cdot  x-  2^{ x}}{  x^{ 2}}\right)$
\end{document}