\documentclass[a4paper,12pt]{article} % добавить leqno в [] для нумерации слева
\usepackage[a4paper,top=1.3cm,bottom=2cm,left=1.5cm,right=1.5cm,marginparwidth=0.75cm]{geometry}
\usepackage{cmap}					% поиск в PDF
\usepackage[warn]{mathtext} 		% русские буквы в фомулах
\usepackage[T2A]{fontenc}			% кодировка
\usepackage[utf8]{inputenc}			% кодировка исходного текста
\usepackage[english,russian]{babel}	% локализация и переносы
\usepackage{physics}
\usepackage{multirow}
\allowdisplaybreaks

%%% Нормальное размещение таблиц (писать [H] в окружении таблицы)
\usepackage{float}
\restylefloat{table}



\usepackage{graphicx}

\usepackage{wrapfig}
\usepackage{tabularx}

\usepackage{hyperref}
\usepackage[rgb]{xcolor}
\hypersetup{
	colorlinks=true,urlcolor=blue
}

\usepackage{pgfplots}
\pgfplotsset{compat=1.9}

%%% Дополнительная работа с математикой
\usepackage{amsmath,amsfonts,amssymb,amsthm,mathtools} % AMS
\usepackage{icomma} % "Умная" запятая: $0,2$ --- число, $0, 2$ --- перечисление

%% Номера формул
\mathtoolsset{showonlyrefs=true} % Показывать номера только у тех формул, на которые есть \eqref{} в тексте.

%% Шрифты
\usepackage{euscript}	 % Шрифт Евклид
\usepackage{mathrsfs} % Красивый матшрифт

%% Свои команды
\DeclareMathOperator{\sgn}{\mathop{sgn}}

%% Перенос знаков в формулах (по Львовскому)
\newcommand*{\hm}[1]{#1\nobreak\discretionary{}
	{\hbox{$\mathsurround=0pt #1$}}{}}

\date{\today}

\usepackage{gensymb}

\begin{document}

\begin{titlepage}
	\begin{center}
		{\large МОСКОВСКИЙ ФИЗИКО-ТЕХНИЧЕСКИЙ ИНСТИТУТ (НАЦИОНАЛЬНЫЙ ИССЛЕДОВАТЕЛЬСКИЙ УНИВЕРСИТЕТ)}
	\end{center}
	\begin{center}
		{\large Физтех-школа радиотехники и компьютерных технологий}
	\end{center}
	
	
	\vspace{4.5cm}
	{\huge
		\begin{center}
			{\bf Отчёт о выполнении лабораторной работы 2.2.8}\\
			Практика взятия производной сложной фукции
		\end{center}
	}
	\vspace{2cm}
	\begin{flushright}
		{\LARGE Автор:\\ Воронцов Амадей Александрович \\
			\vspace{0.2cm}
			Б01-102}
	\end{flushright}
	\vspace{8cm}
	\begin{center}
		Долгопрудный\\
		\today
	\end{center}
\end{titlepage}

\section{Дисклеймер}
Данная статья не рекомендуется к прочтению лиц младше 18 лет, либо не сдавшим коллоквиум студентам первого курса. Приведенная информация является сугубо мнением автора. Данная статья может задеть ваши чувства, связанные с этикой и нормами вашей жизни, поэтому просим всех гуманнитариев убрать глаза от экрана. 

\section{Введение}
\textbf{Цель работы:}  \begin{enumerate}
	\item изучить алгоритм взятия производной сложной функции
	\item бесполезно потратить своё время
\end{enumerate}

\textbf{В работе используются:} ручка, бумага, мозги
\section{Теоретические сведения}
\textbf{Определение:} Производной функции $f$ называется предел $ \lim_{x\to\infty} \frac{f(x) - f(x_0)}{x - x_0}$, если этот предел конечен, либо равен $+ \infty \text{ или} - \infty$. Обозначается этот предел $f'(x_0)$.
\newline
\textbf{Теорема 1:} Пусть функции $f$ и $g$ имеют конечные производные в точке $x_0$. Тогда функции $f + g, f \cdot g, \frac{f}{g}$ имеют производные в точке $x_0.$ (В последнем случае нужно требовать, чтобы $g'(x_0) \neq 0 $), причем в точке $x_0$ выполняются равенства

$$    \left( f(x) + g(x) \right)' = f'(x) + g'(x) $$
$$    \left( f(x)g(x) \right)' = f'(x)g(x) + f(x)g'(x) $$
$$    \left( \frac{f(x)}{g(x)} \right)' = \frac{f'(x)g(x) - f(x)g'(x)}{g^2(x)} $$
\newline
\textbf{Теорема 2:} Пусть функция $f$ имеет конечную производную в точке $x_0$, а функция $g$ имеет конечную производную в точке $u_0 = f(x_0)$. Тогда функция $h(x) = g(f(x))$ имеет производную в точке $x_0$, причем $h'(x_0) = g'(u_0)f'(x_0)$

\section{Ход работы}

\textbf{Задача:} найти 